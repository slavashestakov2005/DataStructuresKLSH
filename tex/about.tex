\coursefooterinfo{Описание}
\head{О курсе}
\subhead{\Large Аннотация}

Хочешь писать ужасные программы на языке C++? Тебе к нам! Если ты уже умеешь хорошо программировать, то наверняка знаешь про разные встроенные структуры данных. На курсе мы разберёмся, как они устроены, и напишем свои реализации на C++. Будет сложно, но интересно. Приходи!

За время курса познакомимся с классами, пространствами имён, указателями и итераторами, шаблонами. Олимпиадное прошлое не требуется, но, тем не менее, в некоторых местах будет достаточно сложно даже олимпиадникам.

Требования: школьники(-цы) 9-10-х классов, знающие C++ (или другой C-подобный язык) на уровне уверенного пользования циклами и встроенными структурами данных (vector, set и т.д.).

\subhead{\Large Результаты}

За курс успели пройти меньше, чем есть в этом файле, а именно:
\begin{itemize}
    \item Из урока 3 пропустили копирование и сравнение объектов;
    \item Из урока 8 пропустили виртуальные методы и приведение типов;
    \item Урок 10 пропустили;
    \item Урок 11 разобрали только теоретически;
    \item Уроки 12 и 13 пропустили;
\end{itemize}

Больше документов по курсу можете посмотреть в \href{https://github.com/slavashestakov2005/DataStructuresKLSH}{\color{blue}{репозитории}}, а по вопросам писать в \href{https://vk.com/id517487657}{\color{blue}{vk}}.
