\coursefooterdate{17.06.2024 -- 06.07.2024}
\head{\Large Урок 3: Доработка Vector.}
\label{md2tex3}
\hyperref[md2texREADME]{\color{cyan}{К главному описанию}}


\subhead{Краткий план}
\begin{enumerate}
    \item Аккуратная работа с памятью (копирование).
    \item Конструктор от размера.
    \item Больше методов: получение размера, удаление из конца; константные методы.
    \item Сравнение объектов: оператор \mintinline{cpp}{<=>}.
    \item Обращение по индексу и итераторы: оператор \mintinline{cpp}{[]}, методы \mintinline{cpp}{begin} и \mintinline{cpp}{end}.
\end{enumerate}


\subhead{Мотивация}
Казалось бы, мы всё уже написали и оно отлично работает. Но вот нет, мы сегодня рассмотрим такие примеры, по которым будет видно наличие ошибок.

\subhead{Копирование}
Рассмотрим вот такой код:
\begin{minted}{cpp}
Vector v1;
v1.push_back(2);
v1.push_back(3);
v1.push_back(5);
v1.push_back(7);
cout << v1;             // 2 3 5 7
Vector v2 = v1;
cout << "v1: " << v1;   // 2 3 5 7
cout << "v2: " << v2;   // 2 3 5 7
v1.push_back(11);
v2.push_back(13);
cout << "v1: " << v1;   // 2 3 5 7 13
cout << "v2: " << v2;   // 2 3 5 7 13
\end{minted}
Хмм, кажется что-то не так! 

Придумано много разных способов обойти такие ошибки, мы рассмотрим \mintinline{cpp}{Swap and Copy}. В нём нужно сделать функции \mintinline{cpp}{swap} и \mintinline{cpp}{copy} и потом уже аккуратно реализовать остальные функции через них:
\begin{minted}{cpp}
void copy(Vector const& v);
void swap(Vector& v);
Vector(Vector const& v);                // конструктор копирования, выражается через copy и swap
Vector& operator=(Vector const& v);     // оператор присваивания, выражается через copy и swap
\end{minted}


\subhead{Больше конструкторов}
Представим, что мы хотим сделать новый конструктор от количества элементов, в таком случае нам нужен:
\begin{minted}{cpp}
Vector(size_t size = 0);
\end{minted}
Где параметр по умолчанию нужно указывать в заголовочном файле, но не нужно указывать в файле с кодом.

Но теперь внимание, "незаметная" разница, у нас будут работать оба варианта:
\begin{minted}{cpp}
Vector v3(2);       // круглые скобки
Vector v4 = 5;      // присваивание
\end{minted}

Второй синтаксис какой-то странный, лучше его явно запретить, для этого придумали слово \mintinline{cpp}{explicit} (запретить неявные преобразования):
\begin{minted}{cpp}
explicit Vector(size_t size = 0);
\end{minted}


\subhead{Больше методов}
Предположим, что нам понадобились методы, говорящие размер вектора, тогда понятно, что хочется написать:

\begin{minted}{cpp}
size_t Vector::size() {
    return cur_size;
}
\end{minted}

Но, представим, что кто-то захотел написать такой код:

\begin{minted}{cpp}
const Vector v3(v2);
cout << "v3: " << v3;
cout << v3.size() << "\n";
\end{minted}

В целом все хорошо, понятно чего хотел автор и видно, что константу изменять он не собирался. Но такой код не скомпилируется, ведь метод \mintinline{cpp}{size} не помечен как константный. Чтобы это работало нужно написать:
\begin{minted}{cpp}
size_t Vector::size() const {
    return cur_size;
}
\end{minted}

Аналогично константными должны быть \mintinline{cpp}{get_back} и \mintinline{cpp}{capacity}.

Теперь реализуем операцию удаления из конца вектора: удаляем (уменьшаем \mintinline{cpp}{cur_size}) и если получилось, что \mintinline{cpp}{4cur_size < cur_capacity}, то выделяем заново память размера \mintinline{cpp}{2cur_size} и копируем туда данные. Почему константы выбраны именно так, мы поговорим на следующем занятии.


\subhead{Сравнение объектов}
В старых стандартах для определения сравнения между объектами, нужно было перегружать все операторы вручную (на самом деле можно написать только оператор \mintinline{cpp}{<} и остальные выразить через него, можете на досуге подумать как именно). В новом же стандарте появился оператор \mintinline{cpp}{<=>}, и можно определить его, оператор \mintinline{cpp}{==} и остальные операторы получатся сами.

Для этого определяем операторы:

\begin{minted}{cpp}
std::strong_ordering operator<=>(Vector const& other) const;
bool operator==(Vector const& other) const;
\end{minted}

Внутри первого можно пользоваться операторами \mintinline{cpp}{<=>} для встроенных типов, или возвращать что-то из списка: \mintinline{cpp}{less} (левое меньше), \mintinline{cpp}{greater} (левое больше), \mintinline{cpp}{equal} (равны). Математики придумали не только привычные по жизни порядки \mintinline{cpp}{std::strong_ordering}, но и некоторые другие, о них можете почитать сами в свободное время.

Функцию равенства можно определять как-то по умному (именно поэтому стандарт заставляет определять её отдельно), но можно и через \mintinline{cpp}{<=>}:
\begin{minted}{cpp}
bool Vector::operator==(Vector const& other) const {
    return (*this <=> other) == std::strong_ordering::equal;
}
\end{minted}


\subhead{Обращение по индексу и итераторы}
Ну все же уже давно хотят получать доступ к элементу вектора по произвольному индексу, давайте научимся!
\begin{minted}{cpp}
int& operator[](int pos);                   // чтобы смочь присвоить значение по индексу
int const& operator[](int pos) const;       // чтобы получить значение по индексу на чтение
\end{minted}

Познакомимся со способом сокращения не удобных названий типов:
\begin{minted}{cpp}
using iterator = int*;                      // теперь вместо int* можно писать iterator
using const_iterator = const int*;          // теперь вместо const int* можно писать const_iterator
\end{minted}

А теперь мы готовы определить функции \mintinline{cpp}{begin} и \mintinline{cpp}{end}, чтобы по нашему вектору можно было делать \mintinline{cpp}{range-based for}:
\begin{minted}{cpp}
iterator begin();                           // возвращаем data
iterator end();                             // возвращаем data + cur_size - указатель на
                                            // элемент следующий после конца
const_iterator begin() const;
const_iterator end() const;
\end{minted}

Теперь оператор вывода можно реализовать без использования \mintinline{cpp}{friend}!
