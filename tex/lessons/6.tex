\coursefooterdate{?.06.2024}
\head{\Large Урок 6: Связный список и итераторы.}
\label{md2tex6}
\hyperref[md2texREADME]{К главному описанию}


\subhead{Краткий план}
\begin{enumerate}
    \item Кто такие итераторы.
    \item Односвязный и двусвязный списки.
\end{enumerate}


\subhead{Мотивация}
Мы уже реализовали почти все линейные структуры данных, которые есть в стандартной библиотеке, теперь дело осталось за малым.


\subhead{Итераторы}
Помнится мы реализовали у Vector методы \mintinline{cpp}{begin} и \mintinline{cpp}{end}, которые возвращали указатели. Но бывают более сложные структуры, у которых переход к следующему элементу не такой тривиальный, как у указателей. Для решения этой проблемы придумали итераторы, которые должны уметь:
\begin{itemize}
    \item Делать операцию \mintinline{cpp}{*} - разыменование итератора, нужно вернуть элемент контейнера.
    \item Делать операцию \mintinline{cpp}{++} - переход итератора вперёд.
    \item Опционально делать операцию \mintinline{cpp}{--} - переход итератора назад.
    \item Делать операцию сравнения \mintinline{cpp}{==} - чтобы проверять, дошёл ли итератор до конца.
    \item Иногда оператор \mintinline{cpp}{<} также имеет смысл.
    \item Определить все псевдонимы (не обязательно, без них тоже будет работать):
\end{itemize}
\begin{minted}{cpp}
using iterator_category = std::forward_iterator_tag;
// также бывают: input_iterator_tag, output_iterator_tag,
// bidirectional_iterator_tag, random_access_iterator_tag, contiguous_iterator_tag
using value_type = T;
using element_type = T;
using pointer = T*;
using reference = T&;
\end{minted}



\subhead{Связные списки}
Концепция связных списков достаточно простая: в каждом элементе будет хранить само значение и ссылку на соседний элемент. Если мы храним ссылку только на одного соседа (следующего) - то это односвязный список, \mintinline{cpp}{forward_list} в C++; а если храним ссылку и на следующий элемент и на предыдущий - то это двусвязный список, \mintinline{cpp}{list} в С++.

Для связных списков нам понадобится определить по три класса: для элемента, итератора и собственно самого связного списка. Приведём здесь класс для элемента и итератора односвязного списка, так как раньше мы такого не писали:
\begin{minted}{cpp}
template<typename T> class ForwardListIterator;
template<typename T> class ForwardList;

template<typename T>
class ForwardListItem {
private:
    T value;
    ForwardListItem<T> *next;
public:
    ForwardListItem(T value, ForwardListItem<T> *next) : value(value), next(next) {}
    friend class ForwardListIterator<T>;
    friend class ForwardList<T>;
};

template<typename T>
class ForwardListIterator {
private:
    ForwardListItem<T> *it;
public:
    ForwardListIterator(ForwardListItem<T> *it = nullptr) : it(it) {}

    T& operator*() {
        return it -> value;
    }

    ForwardListIterator<T> operator++() {
        it = it -> next;
        return *this;
    }

    bool operator==(ForwardListIterator<T> const& other) const {
        if (it == nullptr && other.it == nullptr) return true;
        if (it == nullptr || other.it == nullptr) return false;
        return it -> next == other.it -> next;
    }

    friend class ForwardList<T>;
};
\end{minted}
Сам же односвязный список поддерживает операции вставки и удаления элемента из начала, и по итератору (точнее вставляет или удаляет элемент после переданного итератора).

В двусвязном списке в каждом элементе есть указатели на следующий и предыдущий, поэтому вставку и удаление можно делать с обоих концов, удалять элементы можно по самому итератору, а вставлять как до, так и после переданного итератора. Кроме этого итератор для двусвязного списка должен уметь делать \mintinline{cpp}{--}.

*Замечание: реализация двусвязного списка может оказаться сложнее чем думается :worried:*
