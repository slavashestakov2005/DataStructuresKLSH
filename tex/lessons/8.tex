\coursefooterdate{17.06.2024 -- 06.07.2024}
\head{\Large Урок 8: Rbst, больше наследования.}
\label{md2tex8}
\hyperref[md2texREADME]{\color{cyan}{К главному описанию}}


\subhead{Краткий план}
\begin{enumerate}
    \item Split и Merge.
    \item Рандомизированное бинарное дерево поиск (rbst).
    \item Как написать rbst аккуратно.
    \item Виртуальные методы.
    \item Приведение типов.
    \item Неопределённое поведение.
\end{enumerate}


\subhead{Мотивация}
Настала пора уменьшить глубину наших деревьев поиска, для этого попробуем использовать рандом. Сразу заметим, что глубина любого дерева поиска будет хотя бы логарифм, а значит наша цель - достичь именно его.


\subhead{Split и Merge}
Сегодняшние структуры будут выражаться через две простых операции: \mintinline{cpp}{Split} и \mintinline{cpp}{Merge}. Первая разделяем бинарное дерево поиска по переданному ключа на два дерева поиска (в первом все элементы меньшие, а во втором - больше либо равные). А вторая операция склеивает два бинарных дерева поиска при условии, что ключи одного меньше чем ключи другого.

Через эти две операции можно выразить остальные:
\begin{itemize}
    \item \mintinline{cpp}{find x}: сделаем \mintinline{cpp}{Split} сначала по \mintinline{cpp}{x}, а потом у правой части по \mintinline{cpp}{x + 1} (если тип не поддерживает операцию \mintinline{cpp}{+1}, то придётся определять несколько разных \mintinline{cpp}{Split}). Итого получили три разных куска, средний из них отвечает за \mintinline{cpp}{x}: если он \mintinline{cpp}{nullptr} то таких элементов не было, а иначе - были.
    \item \mintinline{cpp}{insert x}: делаем такие же \mintinline{cpp}{Split} как и в операции \mintinline{cpp}{find} и потом делаем \mintinline{cpp}{Merge} всех частей в порядке \mintinline{cpp}{<x, x, >x}.
    \item \mintinline{cpp}{erase x}: делаем такие же \mintinline{cpp}{Split} как и в \mintinline{cpp}{find}, а оставляем только результат \mintinline{cpp}{Merge} от \mintinline{cpp}{<x} и \mintinline{cpp}{>x}.
\end{itemize}

Обе операции, \mintinline{cpp}{Split} и \mintinline{cpp}{Merge} можно реализовать рекурсивно. Также заметим, что операции \mintinline{cpp}{Split} по сути занимается тем же самым, чем раньше занимался \mintinline{cpp}{find}, следовательно \mintinline{cpp}{Split} можно написать общий для всех деревьев поиска, а менять \mintinline{cpp}{Merge} - в нашем случае на какое-то использование рандома.


\subhead{Рандомизированное бинарное дерево поиска}
В каждой вершине мы будем подсчитывать размер её поддерева. И при операции \mintinline{cpp}{Merge(L, R)} корнем будет становиться корень \mintinline{cpp}{L} с вероятностью $\frac{|L|}{|L| + |R|}$, а \mintinline{cpp}{R} будет корнем с вероятностью $\frac{|R|}{|L| + |R|}$. Можно показать, что тогда итоговая глубина получится логарифмическая, то есть оптимальная.


\subhead{Как это написать}
Во-первых начнём с безжалостных изменений в интерфейсе для бинарного дерева поиск, теперь он превратится в:
\begin{minted}{cpp}
template<typename Child> using bst_t = Child;
template<typename Child> using bst_ptr_t = Child*;          // псевдонимы типов для более понятной записи

template<typename T, typename Child>                        // T - тип значений, Child - тип дерева поиска
class BinarySearchTree {
private:
    void update_info();                                     // вызываем при изменении вершина,
                                                            // она вызывает Child::update
protected:
    T value;
    bst_ptr_t<Child> left, right;
    virtual void update() = 0;                              // объявление функции для дочерних классов
public:
    BinarySearchTree();
    BinarySearchTree(T value);
    BinarySearchTree(T value, bst_ptr_t<Child> l, bst_ptr_t<Child> r);      // конструкторы

    std::pair<bst_ptr_t<Child>, bst_ptr_t<Child>> split_lt_geq(T key);      // разделить на: < x  | >= x
    std::pair<bst_ptr_t<Child>, bst_ptr_t<Child>> split_leq_gt(T key);      // разделить на: <= x | > x

    bst_ptr_t<Child> insert(T value);                 // добавление, используем как rbst = rbst -> insert(3)
    std::pair<bool, bst_ptr_t<Child>> find(T value);  // поиск, используем как rbst = rbst -> insert(3)
    bst_ptr_t<Child> erase(T value);                  // удаление, используем как rbst = rbst -> erase(3)

    virtual ~BinarySearchTree();                      // в деструкторе нужно удалить детей
    void print();                                     // печать, пока не сделали итераторы
};
\end{minted}

А для rbst объявление будет таким:
\begin{minted}{cpp}
template<typename T> class RandomBinarySearchTree;
template<typename T> using rbst_t = bst_t<RandomBinarySearchTree<T>>;
template<typename T> using rbst_ptr_t = bst_ptr_t<RandomBinarySearchTree<T>>;
// псевдонимы типов для более понятной записи

template<typename T>                                                        // T - тип значений
class RandomBinarySearchTree : public BinarySearchTree<T, RandomBinarySearchTree<T>> {
private:
    static size_t get_size(rbst_ptr_t<T> t);                                // получить размер
    static rbst_ptr_t<T> merge(rbst_ptr_t<T> l, rbst_ptr_t<T> r);           // объединение
    size_t cur_size;
    void update() override;                                                 // обновить размер
public:
    RandomBinarySearchTree();
    RandomBinarySearchTree(T value);
    RandomBinarySearchTree(T value, rbst_ptr_t<T> left, rbst_ptr_t<T> right);

    friend BinarySearchTree<T, RandomBinarySearchTree<T>>;                  // чтобы мог вызвать update
};
\end{minted}

С реализацией всё должно быть примерно понятно, достаточно написать по описанию структуры. осталось только пояснить некоторые ключевые слова используемые в объявлениях.

Ключевое слово \mintinline{cpp}{static} в этом контексте означает, что это объявляется метод класса и вызывать его надо будет через \mintinline{cpp}{<Класс>::<Метод>}.


\subhead{Виртуальные методы}
Представим, что у вас есть базовый класс, который хочет, чтобы в дочерних классах была определена какая-то функция. Тогда если объявить её виртуальной (\mintinline{cpp}{virtual}), то её можно будет переопределить в дочерних классах и при её вызове от указателя будет вызываться именно дочерняя функция, а не родительская. Это проявление полиморфизма, и можно привести такой поясняющий пример:
\begin{minted}{cpp}
class Parent {
public:
    void f() const {
        std::cout << "Parent::f" << std::endl;
    }
    virtual void g() const {
        std::cout << "Parent::g" << std::endl;
    }
};
class Child: public Parent {
public:
    void f() const {
        std::cout << "Child::f" << std::endl;
    }
    void g() const override {
        std::cout << "Child::g" << std:: endl;
    }
};

Child c;
Parent* p1 = &c;
p1 -> f();                      // Parent::f
p1 -> g();                      // Child::g
Parent& p2 = c;
p2.f();                         // Parent::f
p2.g();                         // Child::g
\end{minted}

При этом слово \mintinline{cpp}{override} опционально, то есть и без него всё будет работать. Но всё же его лучше указывать, ведь так явно понятно что вы делаете переопределение, а компилятор будет ругаться, если переопределять такое вы не могли. Также кроме этих слов ещё используется \mintinline{cpp}{final}, чтобы показать, что дочерним классам нельзя переопределять функцию.


\subhead{Приведение типов}
При реализации методов bst и rbst нам понадобится несколько раз делать приведение типов, для этого придумано много разных способов:
\begin{itemize}
    \item \mintinline{cpp}{const_cast} - добавление и удаление константности;
    \item \mintinline{cpp}{static_cast} - приведение типов, корректность которых проверяется во время компиляции;
    \item \mintinline{cpp}{dynamic_cast} - приведение типов, можно делать более хитрые преобразование, чем в \mintinline{cpp}{static_cast};
    \item \mintinline{cpp}{reinterpret_cast} - можно делать совсем странные вещи, например строку (из произвольных символов) в число;
    \item \mintinline{cpp}{C-style cast} - в каком-то порядке выбирается приведение типов из списка выше;
\end{itemize}

Нам хватит \mintinline{cpp}{static_cast}, да и в целом его почти всегда достаточно. \mintinline{cpp}{C-style cast} считается плохим синтаксисом и его лучше избегать.


\subhead{Неопределённое поведение}
В текущей реализации есть UB (\mintinline{cpp}{undefined behaviour} - самая не приятная штука в C++) в моменты, когда мы делаем вызовы вида \mintinline{cpp}{nullptr -> split ...}. Но на практике кажется, что функцию вызывать получается и пока не будет разыменовывания нулевого указателя - всё будет работать. Но опять же повторюсь, что это не по стандарту и лучше бы как-то переписать этот фрагмент кода.
