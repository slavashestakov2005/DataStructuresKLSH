\coursefooterdate{17.06.2024 -- 06.07.2024}
\head{Загоночная контрольная}
\def\taskname#1{\underline{\textbf{Задача #1.}}}
\def\taskArow{ & & \\[21 mm] \hline}
\def\taskBrow#1{#1 & & \\[25mm] \hline}
\def\taskCrow{ & \\[15mm] \hline}
\def\tablehead#1{\textbf{#1}}


\taskname{1} Ответьте на перечисленные ниже вопросы и впишите ответы в таблицу.

\begin{itemize}
    \item Перечислите изученные нами структуры данных;
    \item Для каждой структуры кратко напишите зачем она нужна;
    \item Для каждой структуры напишите или нарисуйте как она работает;
    \item Для каждой структуры напишите список поддерживаемых операций;
    \item Для каждой операции напишите её асимптотику;
\end{itemize}

\begin{center}
    \begin{tabular}{|p{58.5mm}|p{67mm}|p{50mm}|}
        \hline
        \tablehead{Структуры и зачем нужна} & \tablehead{Как работает} & \tablehead{Методы и их асимптотика} \\
        \hline
        \taskArow
        \taskArow
        \taskArow
        \taskArow
        \taskArow
        \taskArow
        \taskArow
        \taskArow
    \end{tabular}
\end{center}


\taskname{2} Заполните таблицу про методы доказательства асимптотик.

\begin{center}
    \begin{tabular}{|p{23mm}|p{90mm}|p{62.5mm}|}
        \hline
        \tablehead{Метод доказательства} & \tablehead{Как доказывать} & \tablehead{Где использовали} \\
        \hline
        \taskBrow{Монетки}
        \taskBrow{Потенциалы}
    \end{tabular}
\end{center}


\ \\

\ \\

\taskname{3} Заполните таблицу про синтаксис C++.

\begin{center}
    \begin{tabular}{|p{40mm}|p{140mm}|}
        \hline
        \tablehead{Синтаксис} & \tablehead{Зачем нужен} \\
        \hline
        \taskCrow
        \taskCrow
        \taskCrow
        \taskCrow
        \taskCrow
        \taskCrow
        \taskCrow
        \taskCrow
    \end{tabular}
\end{center}
