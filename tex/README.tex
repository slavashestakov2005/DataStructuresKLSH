\head{\Large План по занятиям:}
\label{md2texREADME}
\begin{enumerate}
    \item \hyperref[md2tex1]{Урок 1} - знакомство с ООП.
    \item \hyperref[md2tex2]{Урок 2} - многофайловые программы, простая реализация Vector.
    \item \hyperref[md2tex3]{Урок 3} - доработка Vector.
    \item \hyperref[md2tex4]{Урок 4} - шаблоны, наследование и структуры на основе Vector.
    \item \hyperref[md2tex5]{Урок 5} - методы монеток и потенциалов.
    \item \hyperref[md2tex6]{Урок 6} - связный список и итераторы.
    \item \hyperref[md2tex7]{Урок 7} - разные варианты дека и интерфейс для бинарных деревьев поиска.
    \item \hyperref[md2tex8]{Урок 8} - rbst, больше наследования.
    \item \hyperref[md2tex9]{Урок 9} - декартово дерево, обычное и по неявному ключу.
    \item \hyperref[md2tex10]{Урок 10} - Splay Tree: амортизированный логарифм в бинарном дереве поиска.
    \item \hyperref[md2tex11]{Урок 11} - 2-3 дерево: строгий логарифм в дереве поиска.
    \item \hyperref[md2tex12]{Урок 12} - пары и set, map, multi- с итераторами.
    \item \hyperref[md2tex13]{Урок 13} - хеширование.
\end{enumerate}
