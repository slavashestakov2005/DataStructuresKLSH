\head{\Large План по занятиям:}
\label{md2texREADME}
\begin{enumerate}
    \item \hyperref[md2tex1]{\color{cyan}{Урок 1}} - знакомство с ООП.
    \item \hyperref[md2tex2]{\color{cyan}{Урок 2}} - многофайловые программы, простая реализация Vector.
    \item \hyperref[md2tex3]{\color{cyan}{Урок 3}} - доработка Vector.
    \item \hyperref[md2tex4]{\color{cyan}{Урок 4}} - шаблоны, наследование и структуры на основе Vector.
    \item \hyperref[md2tex5]{\color{cyan}{Урок 5}} - методы монеток и потенциалов.
    \item \hyperref[md2tex6]{\color{cyan}{Урок 6}} - связный список и итераторы.
    \item \hyperref[md2tex7]{\color{cyan}{Урок 7}} - разные варианты дека и интерфейс для бинарных деревьев поиска.
    \item \hyperref[md2tex8]{\color{cyan}{Урок 8}} - rbst, больше наследования.
    \item \hyperref[md2tex9]{\color{cyan}{Урок 9}} - декартово дерево, обычное и по неявному ключу.
    \item \hyperref[md2tex10]{\color{cyan}{Урок 10}} - Splay Tree: амортизированный логарифм в бинарном дереве поиска.
    \item \hyperref[md2tex11]{\color{cyan}{Урок 11}} - 2-3 дерево: строгий логарифм в дереве поиска.
    \item \hyperref[md2tex12]{\color{cyan}{Урок 12}} - пары и set, map, multi- с итераторами.
    \item \hyperref[md2tex13]{\color{cyan}{Урок 13}} - хеширование.
\end{enumerate}
